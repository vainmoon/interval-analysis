\documentclass[a4paper,14pt]{article}
\usepackage[a4paper, mag=1000, left=2.5cm, right=1cm, top=2cm, bottom=2cm, headsep=0.7cm, footskip=1cm]{geometry}
\usepackage[utf8]{inputenc}
\usepackage[T2A]{fontenc}
\usepackage[english,russian]{babel}
\usepackage{indentfirst}
%\usepackage[dvipsnames]{xcolor}
\usepackage[colorlinks]{hyperref}
\usepackage{amsfonts} 
\usepackage{amsmath}
\usepackage{amssymb}
\usepackage{graphicx}
\usepackage{float}

\DeclareGraphicsExtensions{.png,.jpg}

\usepackage{fancyhdr}
\pagestyle{fancy}
\fancyhead[LE,RO]{\thepage}
\fancyfoot{}

\usepackage{listings}

\hypersetup{linkcolor=black}

\title{non-linear equations}
\author{Умнов Сергей}
\date{2024}
\thispagestyle{empty}
\begin{document}
	
	\begin{titlepage}
		\begin{center}
			\textsc{
				Санкт-Петербургский политехнический университет имени Петра Великого \\[5mm]
				Физико-механический институт\\[2mm]
				Высшая школа прикладной математики и физики            
			}   
			\vfill
			\textbf{\large
				Интервальный анализ\\
				Отчёт по лабораторной работе №1 \\[3mm]
			}                
		\end{center}
		
		\vfill
		\hfill
		\begin{minipage}{0.5\textwidth}
			Выполнил: \\[2mm]   
			Студент: Умнов Сергей \\
			Группа: 5030102/10201\\
		\end{minipage}
		
		\hfill
		\begin{minipage}{0.5\textwidth}
			Принял: \\[2mm]
			к. ф.-м. н., доцент \\   
			Баженов Александр Николаевич
		\end{minipage}
		
		\vfill
		\begin{center}
			Санкт-Петербург \\2024 г.
		\end{center}
	\end{titlepage}
	
	\tableofcontents
	\newpage
	
	\section{Постановка задачи}
	Пусть дана ИСЛАУ

	\[
		Ax = b, \ x = (x_1, x_2)
	\]

	И дана вещественная матрица
	\begin{equation}\text{mid} A = 
		\begin{pmatrix}
			a_{11} & a_{12}\\ 
			a_{21} & a_{22}
		\end{pmatrix}
	\end{equation}

	Рассмотрим матрицу радиусов:
	\begin{equation}
		\text{rad}A = \alpha \begin{pmatrix}
			1 & 1\\ 
			1 & 1
		\end{pmatrix}
	\end{equation}
	Построим интервальную матрицу следующего вида
	\begin{equation}
		A = 
		\begin{pmatrix}
			[a_{11} - \alpha \cdot A^{(1,1)}_i, a_{11} + \alpha \cdot A^{(1,1)}_i ]  & [a_{12} - \alpha \cdot A^{(1,2)}_i, a_{12} + \alpha \cdot A^{(1,2)}_i ]\\ 
			[a_{21} - \alpha \cdot A^{(2,1)}_i, a_{21} + \alpha \cdot A^{(2,1)}_i ] & [a_{22} - \alpha \cdot A^{(2,2)}_i, a_{22} + \alpha \cdot A^{(2,2)}_i ]
		\end{pmatrix}
	\end{equation}
	$i=\overline{1,2}$

	Требуется:
	\begin{itemize}
		\item Найти диапазон значений \( \alpha \), при которых
      \( 0 \in \det A \);
		\item Для минимального значения радиуса матричных элементов
		\( \min \alpha \) найти точечную матрицу \( A' \):
		\[
			\det A' = 0.
		\]
	\end{itemize}
	

	В целях конкретизации и возможности проверки решения будем использовать следующую матрицу
	\begin{equation}
		\text{mid} A = \begin{pmatrix}
			1.05 & 0.95\\ 
			1 & 1
		\end{pmatrix}
	\end{equation}
	\section{Теория}
	Интервалом $[a, b]$ вещественной оси $\mathbb{R}$ называется множество всех чисел, расположенных между заданными числами $a$ и $b$, включая их самих, т. е.
	\begin{equation}
		[a, b] \stackrel{\text{def}}{=}
		\left \{ x \in \mathbb{R} \mid a \leqslant x \leqslant b \right \}.
	\end{equation}
	Основные арифметические операции для интервалов:
	\begin{enumerate}
		\item \textbf{Сложение}
		\begin{equation}
			[a, b] + [c, d] = [a+c, b+d]
		\end{equation}
		\item \textbf{Вычитание}	
		\begin{equation}
			[a, b] - [c, d] = [a-d, b-c]
		\end{equation}
		\item \textbf{Умножение}
		\begin{equation}
			[a, b] \cdot [c, d] = [\min(ac, ad, bc, bd), \max(ac, ad, bc, bd)]
		\end{equation}
		\item \textbf{Деление}
		\begin{equation}
			\frac{[a,b]}{[c,d]}=[\min\Big(\frac{a}{c}, \frac{a}{d}, \frac{b}{c}, \frac{b}{d}\Big), \max\Big(\frac{a}{c}, \frac{a}{d}, \frac{b}{c}, \frac{b}{d}\Big)]
		\end{equation}
	\end{enumerate}

	Характеристики интервалов:
	\begin{enumerate}
		\item \textbf{Средняя точка}
		\begin{equation}
			\text{mid}[a,b] = \frac{1}{2}(a+b)
		 \end{equation}
		\item \textbf{Ширина}
		\begin{equation}
			\text{wid}[a,b] = (b-a)
		\end{equation}
		\item \textbf{Радиус}
		\begin{equation}
			\text{rad}[a,b] = \frac{1}{2}(b-a)
		\end{equation}
	\end{enumerate}
	\section{Реализация}
	Лабораторная работа выполнена на языке программирования Python. В ходе
  работы была использована библиотека \verb!numpy!.
	\subsection{Описание алгоритма}
	Для нахождения минимального значения \(\alpha\) использовался итеративный метод с экспоненциальным увеличением шага:
	\begin{enumerate}
		\item \textbf{Экспонециальный поиск}
		\begin{itemize}
			\item Инициализация: \( k = 0 \), \(\alpha_0 = e^0\).
			\item На каждой итерации значение \( \alpha_k \) обновляется по формуле:
          		\[ \alpha_k = e^k, \quad k_{i+1} = k_{i} + 1. \]
				  \item Процесс продолжается, пока \( 0 \notin \det A(\alpha_k) \), где \( A(\alpha_k) \) - матрица с заданным интервалом.
				  \item Найденное значение \( \alpha_k \) используется в качестве верхней границы \( b_0 \) для следующего этапа.
		\end{itemize}
		\item \textbf{Уточнение методом бисекции}
		\begin{itemize}
			\item Устанавливаются начальные границы: \( a_0 = 0 \), \( b_0 = \alpha_k \)
			\item На каждой итерации вычисляется:
				\[ \alpha_{k+1} = \frac{a_k + b_k}{2}. \]
			\item Если \( 0 \in \det A(\alpha_{k+1}) \), то:
				\[ b_{k+1} = \alpha_{k+1}, \]
			иначе:
				\[ a_{k+1} = \alpha_{k+1}. \]
			\item Процесс продолжается, пока разница \( b - a > \varepsilon \), где \( \varepsilon \) — заданная точность.
			\item После завершения возвращается значение:
				  \[ \alpha^* = \frac{a_k + b_k}{2}. \]	
		\end{itemize}
	\end{enumerate}
	
	\subsection{Ссылка на репозиторий}
	\url{https://github.com/vainmoon/interval-analysis} 
	
	\clearpage

	\section{Результат}

	\subsection{Результаты вычислений параметра регуляризации}

	В процессе итеративного вычисления значения \(\alpha_k\) и детерминанта матрицы \(A_k\) для каждой итерации \(k\) рассчитывались промежуточные значения, которые приведены в таблице ниже. 

	На каждой итерации значение \(\alpha_k\) уточняется с помощью метода бинарного поиска, а детерминант матрицы \(A_k\) вычисляется с использованием соответствующих интервалов.

	\begin{table}[h!]
		\centering
		\begin{tabular}{|c|c|c|}
		\hline
			\textbf{\(k\)} & \textbf{\(\alpha_k\)} & \textbf{\(\det(A_k)\)} \\ \hline
			0 & 0.500000 & \([-1.900000, 2.100000]\) \\ \hline
			1 & 0.250000 & \([-0.900000, 1.100000]\) \\ \hline
			2 & 0.125000 & \([-0.400000, 0.600000]\) \\ \hline
			3 & 0.062500 & \([-0.150000, 0.350000]\) \\ \hline
			4 & 0.031250 & \([-0.025000, 0.225000]\) \\ \hline
			5 & 0.015625 & \([0.037500, 0.162500]\) \\ \hline
			6 & 0.023438 & \([0.006250, 0.193750]\) \\ \hline
			7 & 0.027344 & \([-0.009375, 0.209375]\) \\ \hline
			8 & 0.025391 & \([-0.001563, 0.201563]\) \\ \hline
			9 & 0.024414 & \([0.002344, 0.197656]\) \\ \hline
			10 & 0.024902 & \([0.000391, 0.199609]\) \\ \hline
			11 & 0.025146 & \([-0.000586, 0.200586]\) \\ \hline
			12 & 0.025024 & \([-0.000098, 0.200098]\) \\ \hline
			13 & 0.024963 & \([0.000146, 0.199854]\) \\ \hline
			14 & 0.024994 & \([0.000024, 0.199976]\) \\ \hline
			15 & 0.025009 & \([-0.000037, 0.200037]\) \\ \hline
			... & ... & ... \\ \hline
		\end{tabular}
		\caption{Результаты вычислений для \(\alpha_k\) и \(\det(A_k)\)}
		\label{tab:results}
	\end{table}

	На 12-ой итерации было найдено значение \(\alpha_{min} \approx 0.025\) при заданной точности \(\varepsilon = 10^{-5}\)
	
	
	\subsection{Итоговые результаты}

	Минимальное значение параметра регуляризации:

	\[\alpha_{min} = 0.025\]

	Интервальная матрица имеет такой вид:

	\[
		A = \begin{bmatrix}
		  [1.025,\ 1.075] & [0.925,\ 0.975] \\
		  [0.975,\ 1.025] & [0.975,\ 1.025]
		\end{bmatrix}
	\]

	Определитель этой матрицы имеет вид:

	\[ det(A) = [0.0,\ 0.2] \]

	Диапазон значений параметра регуляризации, при котором определитель интервальной матрицы \(A\) включает ноль:
	
  	\[ \alpha \in [0.025,\ +\infty) \]

	\subsection{Точечная матрица \( A' \)}

	Для найденного минимального значения \( \alpha_{min} \) была определена
	точечная матрица \( A' \), принадлежащая интервальной матрице \( A \),
	такая, что \( \det A' = 0 \).

	Точечная матрица \( A' \):

	\[
		A' = \begin{bmatrix}
		1.025 & 0.975 \\
		1.025 & 0.975
		\end{bmatrix}
	\]

	Строки матрицы являются линейно зависимыми, значит, её определитель равен нулю и матрица является вырожденной.


	\section{Обсуждение}

	\begin{enumerate}
		\item \textbf{Физическая интерпретация} \\
Матрица \( A' \) формируется путем минимизации радиуса матричных элементов \( \delta \), что приводит к необратимости матрицы \( A \), когда её определитель становится равным нулю (\( \det A' = 0 \)). В результате система уравнений становится вырожденной, и её решение теряет однозначность, порождая бесконечное множество возможных решений. С физической точки зрения это указывает на то, что недостаточность данных, полученных с двух ракурсов, не позволяет точно реконструировать объект. Отсутствие необходимой информации для однозначного решения задачи приводит к неопределенности в процессе реконструкции.
		\item \textbf{Чувствительность при минимальном радиусе} \\
Когда радиус матричных элементов достигает минимума, формируется точечная матрица \( A' \), которая является границей множества возможных матриц \( A \), определяющих интервал неопределенности. В таком случае система становится крайне чувствительной к любым незначительным изменениям исходных данных. Даже малейшие искажения или шум в данных могут вызвать существенные изменения в результатах реконструкции, что значительно усложняет решение задачи томографии при работе с реальными данными, содержащими шум. Это подчеркивает критическую важность учета погрешностей данных при выполнении реконструкции.
	
		\item \textbf{Практические соображения} \\
В практической томографии для повышения точности и стабильности решения часто применяют большее количество ракурсов, чем два, что улучшает условия задачи и позволяет избежать ситуации, когда \( \det A' = 0 \). При ограниченном числе ракурсов существует высокая вероятность вырождения матрицы, что делает задачу плохо обусловленной. В таких случаях необходимо использовать специальные методы, учитывающие неопределенность данных и решающие проблему вырождения, такие как методы регуляризации или статистические подходы, которые помогают находить устойчивые решения, несмотря на недостаточность информации.
	\end{enumerate}

	\section{Выводы}
В ходе лабораторной работы была сформирована интервальная матрица \(A\) размером \(2 \times 2\) следующего вида:

\[
A = \begin{bmatrix}
[1.05 - \alpha, 1.05 + \alpha] & [0.95 - \alpha, 0.95 + \alpha] \\
[1 - \alpha, 1 + \alpha] & [1 - \alpha, 1 + \alpha]
\end{bmatrix}
\]

Для этой матрицы был определен диапазон значений \(\alpha\), при которых определитель интервальной матрицы включает ноль, что указывает на вырождение матрицы. Минимальное значение \(\alpha = 0.025\) было установлено с помощью итерационного алгоритма с переменным шагом.

При значении \(\alpha = 0.025\) интервальный определитель матрицы принимает значения в интервале \([0.0, 0.2]\), что включает ноль. Таким образом, это значение \(\alpha\) является минимальным, при котором матрица \(A\) становится вырожденной.

Для минимального значения \(\alpha\) была найдена точечная матрица \(A'\), принадлежащая интервальной матрице \(A\), такая, что:

\[
A' = \begin{bmatrix}
1.025 & 0.975 \\
1.025 & 0.975
\end{bmatrix}
\]

Эта матрица является вырожденной, так как её строки линейно зависимы, и её определитель равен нулю.

В общем случае, если матрица \(A\) представляет собой матрицу линейной регрессии, она может иметь размерность \(2 \times N\) (где \(N \geq 2\)) и не быть квадратной. В таких случаях для анализа необходимо рассматривать всевозможные квадратные подматрицы и для каждой подбирать своё значение \(\alpha\), при котором эти матрицы будут неособенными (невырожденными). Затем, пересечение всех найденных матриц позволяет получить итоговую регуляризованную матрицу, которая удовлетворяет условиям задачи и может быть использована для дальнейшего анализа или вычислений.
